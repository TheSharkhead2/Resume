%%%%%%%%%%%%%%%%%%%%%%%%%%%%%%%%%%%%%%%%%
% Developer CV
% LaTeX Template
% Version 1.0 (28/1/19)
%
% This template originates from:
% http://www.LaTeXTemplates.com
%
% Authors:
% Jan Vorisek (jan@vorisek.me)
% Based on a template by Jan Küster (info@jankuester.com)
% Modified for LaTeX Templates by Vel (vel@LaTeXTemplates.com)
%
% License:
% The MIT License (see included LICENSE file)
%
%%%%%%%%%%%%%%%%%%%%%%%%%%%%%%%%%%%%%%%%%

%----------------------------------------------------------------------------------------
%	PACKAGES AND OTHER DOCUMENT CONFIGURATIONS
%----------------------------------------------------------------------------------------

\documentclass[9pt]{developercv} % Default font size, values from 8-12pt are recommended

\usepackage{enumitem}
%----------------------------------------------------------------------------------------

\begin{document}

%----------------------------------------------------------------------------------------
%	TITLE AND CONTACT INFORMATION
%----------------------------------------------------------------------------------------

% \begin{minipage}[t]{0.375\textwidth} % 45% of the page width for name
% 	\vspace{-\baselineskip} % Required for vertically aligning minipages
	
% 	% If your name is very short, use just one of the lines below
% 	% If your name is very long, reduce the font size or make the minipage wider and reduce the others proportionately
% 	\colorbox{black}{{\HUGE\textcolor{white}{\textbf{\MakeUppercase{Theo}}}}} % First name
	
% 	\colorbox{black}{{\HUGE\textcolor{white}{\textbf{\MakeUppercase{Rode}}}}} % Last name
	
% 	\vspace{6pt}
	
% 	% {\huge Web App Architect} % Career or current job title
% \end{minipage}
% \begin{minipage}[t]{0.35\textwidth} % 27.5% of the page width for the first row of icons
% 	\vspace{-\baselineskip} % Required for vertically aligning minipages
	
% 	% The first parameter is the FontAwesome icon name, the second is the box size and the third is the text
% 	% Other icons can be found by referring to fontawesome.pdf (supplied with the template) and using the word after \fa in the command for the icon you want
% 	\icon{MapMarker}{12}{San Francisco Bay Area}\\
% 	\icon{Phone}{12}{+1 650 575 0625}\\
% 	\icon{At}{12}{\href{mailto:theorodester@gmail.com}{theorodester@gmail.com}}\\	
	
% \end{minipage}
% \begin{minipage}[t]{0.275\textwidth} % 27.5% of the page width for the second row of icons
% 	\vspace{-\baselineskip} % Required for vertically aligning minipages
	
% 	% The first parameter is the FontAwesome icon name, the second is the box size and the third is the text
% 	% Other icons can be found by referring to fontawesome.pdf (supplied with the template) and using the word after \fa in the command for the icon you want
% 	% \icon{Globe}{12}{\href{https://alyx.vance.me}{alyx.vance.me}}\\
% 	\icon{Github}{12}{\href{https://github.com/TheSharkhead2}{TheSharkhead2}}\\
	
% 	\icon{Linkedin}{12}{\href{https://www.linkedin.com/in/theorode/}{linkedin.com/in/theorode/}}\\
% \end{minipage}
% \begin{minipage}[t]{0.5\textwidth} % 27.5% of the page width for the first row of icons
% 	\vspace{-\baselineskip} % Required for vertically aligning minipages
	
% 	% The first parameter is the FontAwesome icon name, the second is the box size and the third is the text
% 	% Other icons can be found by referring to fontawesome.pdf (supplied with the template) and using the word after \fa in the command for the icon you want
% 	\icon{MapMarker}{12}{San Francisco Bay Area}\\
% 	\icon{Phone}{12}{+1 650 575 0625}\\
% 	\icon{At}{12}{\href{mailto:theorodester@gmail.com}{theorodester@gmail.com}}\\	
	
% \end{minipage}
% \begin{minipage}[t]{0.5\textwidth} % 27.5% of the page width for the second row of icons
% 	\vspace{-\baselineskip} % Required for vertically aligning minipages
	
% 	% The first parameter is the FontAwesome icon name, the second is the box size and the third is the text
% 	% Other icons can be found by referring to fontawesome.pdf (supplied with the template) and using the word after \fa in the command for the icon you want
% 	% \icon{Globe}{12}{\href{https://alyx.vance.me}{alyx.vance.me}}\\
% 	\icon{Github}{12}{\href{https://github.com/TheSharkhead2}{TheSharkhead2}}\\
	
% 	\icon{Linkedin}{12}{\href{https://www.linkedin.com/in/theorode/}{linkedin.com/in/theorode/}}\\
% \end{minipage}
% \vspace{0.5cm}


{{\Huge\textcolor{black}{\textbf{{Theo Rode}}}}}

\vspace{0.2cm}
(650) 575 0625 $\quad\bullet\quad$ \href{mailto:theorodester@gmail.com}{theorodester@gmail.com} \\
\href{https://www.linkedin.com/in/theorode/}{linkedin.com/in/theorode} $\quad\bullet\quad$ \href{https://github.com/TheSharkhead2}{github.com/TheSharkhead2}


%----------------------------------------------------------------------------------------
%	PROJECTS
%----------------------------------------------------------------------------------------

\cvsect{Projects}

\begin{entrylist}
	\entry
		{Aug 2022 -\\present}
		{Rust Spotfiy API Wrapper}
		{}
		{\begin{itemize}[noitemsep, topsep=1pt]
			\item Developed an intuitive library structure as a wrapper for the Spotify Web API in Rust. 
			\item Implemented the PKCE extension for OAuth2 authorization with the API. 
			\item Constructed custom formatting algorithms for objects to improve usability of API.
			\item Uploaded as an open source library to Rust's Crates.io package marketplace.
		\end{itemize}}
	\entry
		{Nov 2021 -\\Dec 2021}
		{The Three Body Problem Simulation}
		{}
		{\begin{itemize}[noitemsep, topsep=1pt]
			\item Built a simulation engine in Python, using PyGame, for the Three Body problem: modeling the chaotic motion of three gravitational bodies. 
			\item Collaborated with a group to improve accuracy and dependability of the simulation.
			\item Utilized the simulation to gain intuition for the chaotic system's behavior.
		\end{itemize}}
	\entry
		{Jan 2021 -\\Apr 2021}
		{Music Recommendation Engine}
		{}
		{\begin{itemize}[noitemsep, topsep=1pt]
			\item Built a machine learning model with TensorFlow to predict if a song will be enjoyed.
			\item Interfaced with the Spotify web API to grab song data and formatted it with Pandas. 
			\item Constructed a desktop application to view currently playing Spotify song, rate it to train the model, and see the model's prediction for likeability. 
		\end{itemize}}
	\entry 
		{May 2020 -\\Aug 2020}
		{COVID-19 Data Analysis}
		{}
		{\begin{itemize}[noitemsep, topsep=1pt]
			\item Cleaned and formatted COVID-19 case and death data in Python using Pandas.
			\item Analyzed data using numerous predictive models to determine patterns and create predictions. 
			\item Crafted visual representations of data to demonstrate patterns and predictions. 
		\end{itemize}}
	\entry
		{July 2021 -\\present}
		{Writing and Teaching Math Curriculum}
		{}
		{\begin{itemize}[noitemsep, topsep=1pt]
			\item Identified major issues with traditional middle/high school math education through numerous
			interviews with a diverse group of teachers.
			\item Designed custom course curriculum with class notes typeset in LaTeX.
			\item Organized and taught calculus bootcamp for 9th graders using my custom curriculum.
		\end{itemize}}
\end{entrylist}

%----------------------------------------------------------------------------------------
%	WORK EXPERIENCE
%----------------------------------------------------------------------------------------
\cvsect{Work Experience}
\begin{entrylist}
	\entry 
		{Sep 2022 -\\present} 
		{Homework Hotline Tutor}
		{Harvey Mudd College}
		{\begin{itemize}[noitemsep, topsep=1pt]
			\item Provide math help through a call-in tutoring service free for kids from elementary to high school.
			\item Communicate over phone or messaging, adapting my communication style to the situation.
		\end{itemize}}
\end{entrylist}

%----------------------------------------------------------------------------------------
%	EDUCATION
%----------------------------------------------------------------------------------------

\cvsect{Education}

\begin{entrylist}
	\entry
		{2022-2026}
		{Undergraduate Degree}
		{Harvey Mudd College}
		{\\Current freshman pursuing a degree in Computer Science and Math. Current coursework: CS42 (Principles \& Practice: Comp Sci), MATH19 (Single \& Multivariable Calculus), CHEM42, WRIT1, PHYS23}
	\entry
		{2018-2022}
		{High School}
		{The Nueva School}
		{\\GPA: 3.91 (unweighted). Relavent coursework: Quanutm Information and Computation, Algorithms, Computer Security, Computer Vision, Multivariable Calculus, Linear Algebra, Intro and Advanced Machine Learning, Graph Theory, Game Design, Mobile App Design, Cryptocurrency}
\end{entrylist}



%----------------------------------------------------------------------------------------
%	SKILLS
%----------------------------------------------------------------------------------------
\cvsect{Skills} 
\begin{itemize}[noitemsep, topsep=0pt]
	\item Programming languages: Julia, Rust, Python, Racket, JavaScript, MATLAB, HTML, CSS
	\item Tools: git, TensorFlow, Pandas, React Native, Unreal Engine 
	\item Strong leadership, teaching, and teamwork skills
\end{itemize}

%----------------------------------------------------------------------------------------
%	ADDITIONAL INFORMATION
%----------------------------------------------------------------------------------------
\vspace{0.4cm}
% \cvsect{Hobbies}
% 		I love playing tennis and skiing. I also enjoy sleight of hand magic and have organized a group of fellow magicians to perform at birthday parties.
% 	And I can frequently be found solving puzzles, anything ranging from a rubik's cube to a 5000 piece floor puzzle.  
\begin{minipage}[t]{0.6\textwidth}
	\vspace{-\baselineskip} % Required for vertically aligning minipages
	
	\cvsect{Hobbies}
	I love playing tennis and skiing. I also enjoy sleight of hand magic and have organized a group of fellow magicians to perform at birthday parties.
	Frequently, I can also be found solving puzzles. 
\end{minipage}
\hfill
\begin{minipage}[t]{0.35\textwidth}
	\vspace{-\baselineskip} % Required for vertically aligning minipages
	
	\cvsect{Non profit}
	Volunteered for the Service League of Boys for a year. Serviced bikes for people who needed them and packed food. 
\end{minipage}

%----------------------------------------------------------------------------------------

\end{document}
